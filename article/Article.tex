%
% Complete documentation on the extended LaTeX markup used for Insight
% documentation is available in ``Documenting Insight'', which is part
% of the standard documentation for Insight.  It may be found online
% at:
%
%     http://www.itk.org/

\documentclass{InsightArticle}


%%%%%%%%%%%%%%%%%%%%%%%%%%%%%%%%%%%%%%%%%%%%%%%%%%%%%%%%%%%%%%%%%%
%
%  hyperref should be the last package to be loaded.
%
%%%%%%%%%%%%%%%%%%%%%%%%%%%%%%%%%%%%%%%%%%%%%%%%%%%%%%%%%%%%%%%%%%
\usepackage[dvips,
bookmarks,
bookmarksopen,
backref,
colorlinks,linkcolor={blue},citecolor={blue},urlcolor={blue},
]{hyperref}
% to be able to use options in graphics
\usepackage{graphicx}
% for pseudo code
\usepackage{listings}
% subfigures
\usepackage{subfigure}


%  This is a template for Papers to the Insight Journal. 
%  It is comparable to a technical report format.

% The title should be descriptive enough for people to be able to find
% the relevant document. 
\title{A consolidated morphology architecture, including decomposable and recursive implementations}

% Increment the release number whenever significant changes are made.
% The author and/or editor can define 'significant' however they like.
\release{0.00}

% At minimum, give your name and an email address.  You can include a
% snail-mail address if you like.
\author{Richard Beare{$^1$} {\small{and}} Ga\"etan Lehmann{$^2$}}
\authoraddress{{$^1$}Department of Medicine, Monash University, Australia.\\
{$^2$}Unit\'e de Biologie du D\'eveloppement et de la Reproduction, Institut National de la Recherche Agronomique, 78350 Jouy-en-Josas, France}


\begin{document}
\maketitle

\ifhtml
\chapter*{Front Matter\label{front}}
\fi


\begin{abstract}
\noindent
The mathematical morphology operations of erosion, dilation, opening
and closing are members of an important family of non-linear which
compute rank statistics of a kernel. This paper introduces a number of
well known techniques that can be used to dramatically improve the
execution times and describes a new, consolidated class structure via
which the various approaches can be accessed.
\end{abstract}

\tableofcontents


\section{Introduction}
An erosion operation may be loosely defined as replacing the origin of
a kernel by the minimum value in the region defined by the kernel. The
ITK erosion filter, {\em GrayscaleErodeImageFilter}, performs this
operation directly using a neighborhood iterator. While this is quite
a neat example of the use of neighborhood iterators and boundary
conditions, it the worst possible way of implementing an erosion,
resulting in a complexity proportional to the number of pixels in the
kernel. This complexity is especially significant when dealing with
large structuring elements and tends to limit the applicability of
these otherwise very useful filters.

This paper will describe two classes of technique that dramatically
improve the performance of these operations. One technique is
applicable to arbitrary shaped structuring elements and reduces the
complexity from $O(n^d)$ to $O(n^{d-1})$, where $n$ is the kernel size
and $d$ is the kernel dimension. The other is applicable to only
certain shapes of structuring element and produce a complexity that is
{\bf independent} of kernel size.

\section{Flat vs non-flat structuring elements}
Morphological operations may involve flat or non-flat structuring
elements. Non-flat structuring elements are less commonly used than
flat, and are analogous to a weighted kernel. This discussion will
focus on flat structuring elements, although technique described in
Section \ref{sect:movingHist} is applicable to non-flat structuring
elements.

\section{General background}
There are, broadly speaking, two approaches to speeding up kernel
based filters. The first is to exploit properties such as
separability, which may allow an n-dimensional filter to be
implemented as several one dimensional filters. When this sort of
decomposition is possible complexity is typically reduced from
$O(n^d)$ to $O(nd)$. The second approach is to exploit redundancy in
the computation to produce a recursive implementation of the filter,
meaning that the result for a pixel becomes dependent on the result
for neighboring pixel. In the best case scenario the complexity may
become independent of kernel size.

The gaussian convolution filters in ITK exploit both approaches.

\section{Arbitary structuring elements}
\label{sect:movingHist} 
The technique used to implement morhpological operations by arbitary
structuring is based on a updatable histogram and was described in
\cite{Vandroogenbroeck96.3}. The basic concept is very simple. 
A histogram is computed for the kernel centred on the first pixel of
the image. The maximum or minimum pixel value can easily be computed
using the histogram. The histogram for the neighboring pixel can be
computed by discarding outgoing pixels and adding incoming
pixels. Outgoing and incoming pixels for a particular kernel can be
automatically computed when the kernel is created.

This is an example of a recursive implementation.

\subsection{Differences between this implementation and original paper}
The original paper restricted the pixel type to 8 bit so that a 256
element histogram could be used. This implementation uses a C++ map
for larger pixel types and a fixed size type for 8 and 16 bit pixels.

The original paper also advocated sliding the kernel in a zigzag
pattern through the image. Our experiments showed that sliding the
kernel in raster order, which required saving the histogram between
lines, offered a performance advantage (approximately 15\%) for large
images. This is presumably related to cache activity.

\section{Decomposable structuring elements}
A variety of useful structuring element shapes can be composed from
line segments. The classical example is a rectangle - an erosion by a
$n \times m$ rectangle can be produced by carrying out an erosion by a
horizontal line of length $n$ followed by an erosion by a vertical
line of length $m$. Circles, or at least useful approximations of
circles, can be produced using a radial decomposition
\cite{Adams93}. Useful approximations of spheres, based on platonic
solids and other shapes, can also be constructed this way.

The ability to decompose structuring elements into line operations is
especially significant because there are a number of very fast,
recursive, algorithms for erosions and dilations that have complexity
that is independent of line length. The most famous of these are
probably \cite{vanHerk1992a,Gil1993}, which require approximately 3
operations per pixel. A more recent advance, using {\em anchors}
\cite{Vandroogenbroeck2005Morphological} reduces this 
further\footnote{This discussion is very brief. More details of
existing algorithms are available in
\cite{Vandroogenbroeck2005Morphological}}.

The implementation of anchor based morphology extends the algorithm
described in \cite{Vandroogenbroeck2005Morphological} to support
arbitary pixel types. Lines at arbitary angles are also supported
using the techniques described in \cite{soille-breen-jones96}.

\section{The new class heirarchy}

\appendix



\bibliographystyle{plain}
\bibliography{InsightJournal,local}
\nocite{ITKSoftwareGuide}

\end{document}

